\section{Introduction to HTTP}

In 1996, the Hyper Text Transfer Protocol (HTTP) version 1.0 was standardised by Tim Berners-Lee and the Internet Engineering Task Force (IETF). This protocol defines the precise way in which a `client' (which is usually a web browser, such as Internet Explorer) can request resources stored on a remote computer known as a `server'. Clients open a connection, send a `request' for a given resource (identified by a URL), and the server sends a `response' containing the requested resource. From the perspective of the HTTP protocol itself, the type of resource that is transferred is arbitrary: it could be a text file, an image file, or even some dynamically generated content such as the current time. HTTP is one of many network protocols that is used to transmit data between two computers, however when a browser is requesting a resource from a server this is done exclusively over HTTP\@. 

The IETF issued a minor revision of the HTTP/1.0 specification, publishing the HTTP/1.1 specification in 1999~\cite{h1}. In 2015, the IETF finalised and published the HTTP/2 specification~\cite{h2}, which is a major revision of the HTTP protocol. As of January 2018, out of 2.8 million of the most popular websites, 5\% fully supported HTTP/2~\cite{isthewebhttp2yet}. All major desktop browsers and most mobile browsers now support HTTP/2~\cite{caniuseh2}.

This project will explore HTTP/2 and evaluate the benefits that HTTP/2 offers in comparison to HTTP/1.1 for both users and server operators. To assist me in this, I have programmed a web server from scratch that supports HTTP/2 which I will be referencing and using for  demonstration purposes. The source code for the server can be found at~\cite{hh}.
